% !TEX TS-program = XeLaTeX
% !TEX encoding = UTF-8 Unicode

%%%%%%%%%%%%%%%%%%%%%%%%%%%%%%%%%%%%%%%%%%%%%%%%%%%%%%%%%%%%%%%%%%%%%
%
%	大连理工大学硕士论文 XeLaTeX 模版 —— 宏包配置文件 packages.tex
%	版本:0.6
%	最后更新:2010.11.15
%	修改者:Yuri (E-mail: yuri_1985@163.com)
%	编译环境:Ubuntu 10.04 + TeXLive 2010 + TeXworks
%
%%%%%%%%%%%%%%%%%%%%%%%%%%%%%%%%%%%%%%%%%%%%%%%%%%%%%%%%%%%%%%%%%%%%%

% 页面设置
\usepackage[body={16.1cm, 22.2cm}]{geometry}
\usepackage{indentfirst}   % 首行缩进宏包
\usepackage[sf]{titlesec}	% 控制标题的宏包
\usepackage{titletoc}		% 控制目录的宏包
\usepackage{fancyhdr}		% 自定义页眉页脚
\usepackage{fancyref}		% 引用链接属性
\usepackage[perpage,symbol]{footmisc}	% 脚注控制
\usepackage{cite}			% 支持引用的宏包
\usepackage{layouts}		% 打印当前页面格式的宏包
\usepackage{paralist}		% 一种换行不缩进的列表格式,asparaenum,inparaenum 等
\usepackage[numbers]{natbib}	% 参考文献
\usepackage{fancyvrb}		% 原样输出
\usepackage[amsmath,thmmarks,hyperref]{ntheorem} % 定理类环境宏包
\usepackage{type1cm}    % 控制字体的大小


% 图形相关
\usepackage{graphicx}	% 请在引用图片时务必给出后缀名
\usepackage{subfigure}		% 插入子图形
\usepackage{color}		% 支持彩色
\usepackage[below]{placeins}	% 浮动图形控制宏包
%\usepackage{float}       % 不让图片乱跑
\usepackage{rotating}		% 图形和表格的控制
\usepackage{multirow}     % 占多行,复杂点的表格需要
\usepackage{booktabs}     %画三线表
\usepackage{tabularx}     % 占满行,表格
\usepackage[subfigure]{ccaption}	% 插图表格的双语标题
\usepackage{setspace}		% 定制表格和图形的多行标题行距
\usepackage{listings}      %贴代码
\usepackage{extarrows}    %箭头符号
% 其他
\usepackage{calc}   % 在 tex 文件中具有一些计算功能,主要用在页面控制。
\usepackage[xetex,
            bookmarksnumbered=true,
            bookmarksopen=true,
            colorlinks=true,
            %pdfborder={0 0 1},
            citecolor=blue,
            linkcolor=blue,
            anchorcolor=green,
            urlcolor=magenta,
            breaklinks=true,
            ]{hyperref}













