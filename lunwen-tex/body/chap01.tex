% !TEX TS-program = XeLaTeX
% !TEX encoding = UTF-8 Unicode

\chapter{盲信号处理理论基础}
\label{chap01}

\section{盲信号处理概述}
盲信号处理问题可以粗略地表述为:在一个多输入多输出(MIMO)
系统中,当传输信道特性未知或所知甚少时,从输出端的传感器阵列
或转换器输出信号中,分离或估计出源信号的波形,再或者辨识
系统的数学模型。盲信号处理是随着数字通信和地球物理勘探等行业
的飞速发展,在信号处理领域兴起的一个新的研究方向。其主要任务是
对未知系统,在其输入信号完全未知或者只有很少先验知识的情况下,
仅有其输出信号来重构输入信号或进行系统辨识。

从这个粗略的定义可以看到,盲信号处理与传统信号处理的最大区别
在于它试图利用最少的信息获取最大的收益,它与神经网络中的无师学习
和模式识别中的动态聚类有着紧密的联系。在天线阵列处理、语音增强
和分离、数字通信中码间干扰和多径效应的消除、地球物理勘探建模
(地震反卷积)、超声分析中都有着广泛的用途。(?*?*?)

因为是在缺乏混合系统和/或滤波过程参数的情况下进行源信号参数
估计的,所以盲信号处理看起来似乎有点不可思议,也很难想像能将
源信号完全估计出来。事实上,在缺乏某些先验知识时,是不可能唯一地
确定源信号的,然而在允许一定程度上的不确定性时,对源信号加以估计
则通常是可能的。用数学术语来说,这种不确定性和模糊性可以看做是
对被估计的源信号的任意比例的伸缩、排序或时滞,但它依然保留了
源信号的波形信息。尽管这种不确定性使盲信号处理具有一定的局限性,
但在很多实际应用中它并非关键问题,因为源的大量相关信息蕴含在
源信号的波形中,而不是信号的振幅或者系统输出的排列顺序中。(***)

盲信号处理的问题分为盲辨识、盲反卷积、盲信号分离三类。

\subsection{盲辨识}
盲辨识是仅利用系统的带噪输出来辨识系统数学模型的盲信号处理方法。
盲辨识问题与传统系统辨识问题存在的一个最大的不同是,在传统系统辨识
问题中,通常假设系统输入和系统输出均可进行测量,而盲辨识中系统输入
是不可测量的,这就要求利用新的信号处理方法对系统的带噪输出进行处理,
得到其它信息,以补偿系统输入未知这一缺陷。

通常使用的方法是过采样技术和高阶统计信息理论。
\paragraph*{过采样技术}
是通过对系统输出进行过采样后,输出具有循环平稳特性的信号,并包含着
系统的相位信息,从而为系统模型的建立提供了可能。(***)

20世纪80年代初,Garder发现利用调制信号的循环平稳特性,可应用二阶
统计理论对通信信道的幅值和相位进行恢复(?*?*?),从而引起了人们采用
二阶统计理论对系统进行盲辨识研究的热潮。其实现方法是采用过采样技术,
即在输入信号的一个采样周期内,系统输出被采样若干次,这样得到的输出
信号不但具有循环平稳特性,而包含着系统的相位信息,从而补偿了系统输入
信号未知这一缺陷,可对系统进行盲辨识。过采样技术具有运算量适中,
算法收敛性好,可辨识非最小相位系统等优点,从而引发了人们的广泛关注。(***)

自从Garder提出可应用而阶统计理论对通信信道的幅值和相位进行恢复以来,
过采样技术取得了很大的发展,目前的研究主要集中在有限冲击响应系统
脉冲序列的确定及算法的可辨识行方面。(***)

\paragraph*{高阶统计信息理论}是二阶相关概念的一种自然推广,
是按照一定的方式对观测量进行多次相关。
20世纪80年代初期,Li和Rosenblatt运用高阶统计信息理论
对非高斯系统进行了研究分析,提出了对各类非高斯问题的统一描述,(?*?*?) 
从而使高阶统计信息理论在将近二十年来得到了突飞猛进的发展。
高阶统计信息理论不但能对系统进行辨识,
而且可用于估计系统的相位,即可以用于非最小相位系统的辨识。
其主要缺点是计算累积误差大,数据窗口长,数值计算量大,
且仅适用于非高斯系统。

盲辨识问题时一个极具挑战性的研究课题,牵涉的知识面广。应用
过采样技术对系统进行盲辨识缺乏系统的理论和方法,还处于发展阶段。
而高阶统计信息理论经过多年的发展,取得了很多有益的成果,但主要还
停留在理论研究方面,工程实践应用还比较少。(?*?*?)

\subsection{盲反卷积(***)}
盲反卷积时仅利用系统的带噪输出对系统的输入进行重构,一般假定
系统输入u(t)为独立、同分布、零均值序列,系统模型未知,系统的输出
可以进行测量。
%          ______________                  ________
%输入u(t)  |             |系统带噪输出y(t) |       | ὒ(t)
%———————-->|未知系统模型 |---------------->|滤波器 |-------->
%          |_____________|                 |_______|

    更一般地,传感器测得的信号是源机器滤波和延迟的混叠信号的线性
组合,通常称为卷积混叠。无噪声多通道卷积混叠信号的数学模型可以
用下式表示:
\begin{equation} 
\bm{x}(t)=\bm{A}(z)\bm{s}(t) = 
\sum^\infty_{k=-\infty}\bm{A}(k)\bm{s}(t-k)
\end{equation}
式中,$\bm{x}(t)$和$\bm{s}(t)$同为瞬时混叠式;$\bm{A}(k)$为未知滤波混叠矩阵,
$\bm{A}(z)$为其Z变换;观测信号$\bm{x}(t)$是源信号$\bm{s}(t)$通过$\bm{A}(k)$的卷积混合,
所以矩阵序列$\{\bm{A}(k)\}$又称冲激响应。
    
	现有的多通道盲反卷积方法大都是仅通过观测信号$\bm{x}(t)$估计通道
冲击响应$\{\bm{A}(k)\}$,进而恢复源信号。我们可以直接给出盲反卷积(盲均衡)
模型:
\begin{equation}
\bm{y}(t) = \sum^\infty_{k=-\infty}\bm{W}(k)\bm{x}(t-k)=\bm{W}(z)\bm{x}(t)
\end{equation}
其中,$\bm{y}(t)$为均衡输出矢量;$\bm{W}(z)$称为均衡器,$\bm{W}(k)$为均衡器系数矩阵。

实际进行盲反卷积的算法有:
\paragraph*{H-J算法的扩展(***)}
这类算法是瞬时混叠盲源分离H-J网络训练算法的直接推广。针对反馈
分离网络和卷积混叠模型,卷积混叠信号的反馈分离网络输出为
\begin{equation}
\bm{y}(t)=\bm{x}-\sum^{L}_{k=0}\bm{W}(k)\bm{y}(t-k)
\end{equation}

Platt和Faggin为这类网络确定了用来进行优化的理论准则————最小
输出功率原理,即当独立分量得以分离,则信号的功率达到最小,由此
利用梯度下降法得到该网络的训练公式,它恰好是H-J算法的推广,公式为:
\begin{equation}
\Delta\omega_{ij}(k)=a y_i(t)y_j(t)
\end{equation}
\paragraph*{累积量算法的扩展(***)}
这是一种基于互累积两消失的自适应盲反卷积算法。该算法利用了
盲源分离的经验公式,直接将盲源分离的情况扩展到盲反卷积。公式在形式
上和扩展的H-J方法类似,而H-J公式中的非现行函数用四阶互累积量
$c_{31}(y_i^3(t)y_j(t))$ 和 $c_{31}(y_i(t)y_j^3(t))$
代替,训练公式为:
\begin{equation}
\Delta\omega_{ij}(k)=-ac_{31}(y_i(t)y_j(t-k))
\end{equation}
\paragraph*{信息理论算法扩展(***)}
对瞬时混叠信号模型,Infomax方法是较成功的盲源分离方法之一。
Torkkola将Infomax算法推广到卷积混叠的情况,得到了一个局部训练
算法(仅有两个源的情况)。该算法通过因果滤波器最大化输出熵最小化
两个输出之间的互信息,基于最大信息传输原理,可以得出直接滤波器
零延迟权系数,直接滤波非零延迟权系数和反馈交叉滤波权系数的训练
公式。而对于更一般的卷积混叠情况,Lee给出了反卷积系统滤波器权系数
的公式:
\begin{eqnarray}
\Delta\bm{W}(0)	= \alpha [\bm{(I+uy^T)W}(0)] \\
\Delta\bm{W}(k)	= \beta [\bm{uy^T}(t-k)\bm{W}(k)]
\end{eqnarray}
Lambert在他的博士论文中,发展了FIR滤波器多项式代数理论,
并通过仿真实例表明该理论是解决多通道盲源分离的一种有效工具。
应用FIR多项式矩阵代数的基本思路是扩展标量矩阵代数到时域的
滤波器矩阵代数(或频域的多项式矩阵)。利用这一理论可以得到频域
扩展的Infomax算法:
\begin{eqnarray}
\Delta\bm{W} = \alpha [\bm{W}^{-H}+FFT(\bm{u})\bm{X}^H]
\end{eqnarray}
其中,$H$表示复共轭,$\bm{W}$为滤波器矩阵,$\bm{X}$和$\bm{u}$
为频率域多传感器信号块。

\subsection{盲源分离}
所谓盲信号分离,亦即盲源分离,是在没有关于混合矩阵的任何先验
知识的情况下,来辨识混合矩阵或回复所有源信号,它撇开了混合矩阵A的
特性,而只利用接受信号所携带的信息,这种盲方法的性能在本质上将不受
传输模型误差或阵列误差的影响。

盲源分离和所谓独立分量分析(Independent Component Analysis, ICA)
有着非常密切的联系,很多时候,两个名称甚至混用,关于此部分内容,
将在下一章详细介绍。
































\section{盲信号处理历史和现状}
盲源分离(Blind Source Separation, BSS)是指在不知道源信号
和传输通道的参数的情况下,根据输入源信号的统计特性,仅由
观测信号恢复出源信号各个独立成分的过程。现在所指的
盲源分离通常是对观测到的源信号和的线性瞬时混叠信号进行
分离。(***)

在考虑到时间延迟的情况下,观测到的信号应该是源信号和通道
的卷积,对卷积混叠信号进行盲分离通常称为
盲反卷积(Blind Deconvolution, BD) (***)
(张发启 等 盲信号处理及应用 P1)

对含有未知分布噪声的混叠信号进行盲源分离在当前仍然有困难,
上述两种方法现在一般都没有考虑其影响。除了上述的线性瞬时
混叠和卷积混叠外,实际情况中,更一般的情况是非线性的混叠。

较早进行盲源分离方法研究的是Herault和Jutten,他们提出了
一种类神经元分离的方法:基于反馈神经网络,通过选取奇次
的非线性函数构成Hebb训练,从而达到盲源分离的目的(?*?*?)
该方法不能完成多于两个混叠信号的分离,这是因为非线性函数
的选取具有随意性,并且缺乏理论解释。(?*?*?)

Tong和Liu分析了盲源分离问题的可分离性和不确定性,并给出
了一类基于高阶统计的矩阵代数特征分解方法。(?*?*?)
Cardoso提出了基于高阶统计的联合对角化盲源分离方法,
并应用于波束形成。(?*?*?)
Common系统地分析了瞬时混叠信号的盲源分离问题,并明确了
独立分量分析(Independent Component Analysis, ICA)
的概念(?*?*?),利用可以测度源信号统计独立性的Kullbak-
Leibler准则作为对比函数,通过对概率密度函数的高阶近似,
得出用于测度信号各分量统计独立的对比函数,并由此给出一类
基于特征分解的独立分量分析方法。(?*?*?)
Sejnowski和Bell基于信息理论,通过最大化输出非线性节点
的熵,得出一种最大信息传输的准则函数,并由此导出一种
自适应盲源分离和盲反卷积方法(Infomax),当该方法中
非线性函数的选取逼近源信号的概率分布时可以较好地恢复出
源信号。该算法只能用于源信号峭度(Kurtosis)大于某一值的信号
的盲分离,所以它对分离线性混叠的语音信号非常有效。(?*?*?)
Amari和Cichocki基于信息理论中概率密度的Gram Charlier
展开利用最小互信息(Minimum Mutual Information, MMI)准则函数
得出一类前馈网络的训练算法,可以有效分离具有负峭度的源信号,
该类算法具有等变(equivariant)特性,即不受混叠矩阵的影响。(?*?*?)
Hyvӓrinen基于源信号非高斯性测度(或峭度),给出了一类
定点(Fix-point)训练算法,该类算法可以提取单个具有正或负
峭度的源信号。(?*?*?)

在对线性瞬时混叠信号盲源分离方法进行研究的同时,人们对
卷积混叠信号盲分离————盲反卷积方法也进行了研究。
Platt和Faggin将H-J算法推广到具有时间延迟的卷积混叠
情况。(?*?*?)
Yellin和Wensten给出了基于高阶累积量和高阶谱多通道
盲反卷积方法,通过递归特征分解可以同时进行盲系统参数辨识
和盲反卷积。由于用到高阶累积量和需计算高阶谱,该方法
所需计算量极大。(?*?*?)
Thi和Jutten同样利用四阶累积量或四阶矩函数,给出了
卷积混叠信号盲分离的自适应训练方法。(?*?*?)
K. Tokkola 提出了一个反馈网络结构,将Infomax算法
推广到更广泛的情况,即具有时间延迟的源的混叠或卷及混叠
信号的盲分离。(?*?*?)
Lee和Bell将基于信息最大传输或最大似然算法得出的
盲源分离算法变换到频率域,并利用FIR多项式技术进行
盲反卷积。(?*?*?)

最近人们已经开始研究存在噪声的混叠和非线性混叠信号
信号的盲分离问题。非线性盲分离比线性情况的分离难度更大。
较早涉及非线性混叠信号盲分离的是Burel,他用一个两层
感知器和基于误差向后传输思想的无监督训练算法,通过梯度
下降算法优化统计独立的测度函数,得到一种盲分离算法。
该算法可以用于非线性混叠信号的盲分离。(?*?*?)
1996年Parra提出一类前向信息保持非线性结构映射网络,
通过最小化互信息,减小输出各分量间的剩余度,从而可以
得到非线性独立成分。(?*?*?)
Pajunen, Hyvӓrinen和Karhunen用
自组织映射(Self Organization Map, SOM)从非线性混叠信号
中恢复源信号,该算法可以不考虑非线性混叠的形式,但其网络
复杂性呈指数增长且在分离连续源时存在严重的插值误差。(?*?*?)
Yang和Amri利用两层感知器网络结构,通过最大熵和最小互信息
作为刻度独立的代价函数,提出了信息向后传输的训练方法。
当合理选择非线性函数时,该算法可以分离出一些特定
非线性混叠的源信号。(?*?*?)
Taleb和Jutten提出了一种非线性混叠信号盲分离算法,可以
对被称为后非线性混叠的信号进行盲分离。
由于存在噪声的信号分离是困难的,因此以上方法都没有考虑
噪声的影响。因此有人把带噪声的信号混叠看做一种非线性的,
所以现有的一些带噪声混叠的盲分离方法都是利用非线性方法实现的。
Moulines和Cardoso利用逼近最大似然方法进行带噪声混叠信号
的盲分离和盲反卷积,其中用于处理不完全数据的
期望最大化(Expectation Maximinzing, EM)方法是主要的
数学工具。(?*?*?)
Hyvӓrinen指出,在混叠过程中存在噪声意味着观测数据和源
信号存在非线性关系,他用了独立成分和混叠矩阵的联合最大似然
估计方法。(?*?*?)

在对盲源分离理论上的研究已经取得一定进展之后,人们开始
研究盲源分离训练方法的实际应用。
Lee和Bell将基于信息最大传输或最大似然算法得出的盲源分离
训练算法进行盲反卷积,并用于真实记录的语音信号分离。实验证明,
分离后的语音识别率得到了提高。(?*?*?)
Karhunen和Hyvӓrinen等讲神经网络盲分离算法用于提取图像
特征和分离医学脑电波信号。(?*?*?)
Makeig, Jung和Bell等用盲源分离方法将从
脑电(electro-encephalo-graphic, EGG)信号中记录的与事件相关
的相应数据分解为与传感器数量相等的成分。(?*?*?)
Mackeown等还将ICA用于分析核磁共振成像数据集。(?*?*?)
Sahlin和Broman在移动通信的手机中增加一概麦克风,用信号
分离算法改善通信中信号传输之前的信噪比。(?*?*?)
	
国内近期关于盲信号处理理论和应用技术的研究几乎是与国际上
同步的。
凌燮亭利用反馈式神经网络根据Hebbian的学习算法,实现了
近场情况下一般信号的盲分离,并对算法的渐进收敛性和实现信号
分离状态的稳定性进行了讨论。(?*?*?)
何振亚在基于特征分析、高阶谱的盲源分离鹤盲反卷积方法研究中,
提出了一系列新的基于高阶统计和信息理论的判据和算法。在盲系统
参数估计和盲波束形成等方面也取得了很多很好的研究成果。(?*?*?)
最近胡光锐也开始盲语音分离的研究,并提出了基于高斯混合模型
概率密度估计的语音分离方法。

