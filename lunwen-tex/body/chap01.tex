% !TEX TS-program = XeLaTeX
% !TEX encoding = UTF-8 Unicode

\chapter{模版使用说明}
\label{chap01}

\section{个人信息}
使用模版的第一步当然是修改您的个人信息。与个人信息有关的内容位于~{/preface/cover.tex}~文件中。
对照着模版内容改就好了,没有什么难度。填写专业、姓名和导师的时候注意添加适当空格,也就是$\scriptsize{\sim}$字符,以保持段落对齐。
这里的默认完成时间是最后一次编译main.tex的日期。

\section{中英文摘要、关键字}
中英文摘要和关键字也位于~{/preface/cover.tex}~文件中,分别定义在cabstract, eabstract, ckeywords, ekeywords中,替换成自己的即可。

这里附上研究生院对摘要和关键字的要求:
\begin{asparaenum}
\item “摘要”是摘要部分的标题,不可省略。论文摘要是学位论文的缩影,文字要简练、明确。内容要包括目的、方法、结果和结论。单位制一律换算成国际标准计量单位制,除特殊情况外,数字一律用阿拉伯数码。文中不允许出现插图,重要的表格可以写入;
\item 关键词请尽量用《汉语主题词表》等词表提供的规范词。关键词之间用全角分号间隔,末尾不加标点;
\item 英文摘要和中文摘要对应,但不要逐字翻译。英文关键字使用半角分号间隔,末尾同样不加标点。
\end{asparaenum}

\section{正文}
正文部分包括了引言(chap00.tex)、正文内容章节(chap01.tex、chap02.tex、……)、结论(conclusion.tex)三个部分,均位于body文件夹中。同时位于body文件夹下的还有Bib\TeX{}参考文献文件(reference.bib)。

正文内容章节以chapXX.tex形式为文件名,从01开始计数,使得文件名序号即为章节序号。这些正文内容章节需要依次写入main.tex文件中,格式为
~\texttt{\footnotesize \textbackslash include\{body / chapXX\}}。

同样,附上研究生院对正文的要求:

“正文”不可省略。

正文是硕士学位论文的主体,要着重反映研究生自己的工作,要突出新的见解,例如新思想、新观点、新规律、新研究方法、新结果等。正文一般可包括:理论分析;试验装置和测试方法;对试验结果的分析讨论及理论计算结果的比较等。

正文要求论点正确,推理严谨,数据可靠,文字精练,条理分明,文字图表清晰整齐,计算单位采用国务院颁布的《统一公制计量单位中文名称方案》中规定和名称。各类单位、符号必须在论文中统一使用,外文字母必须注意大小写,正斜体。简化字采用正式公布过的,不能自造和误写。利用别人研究成果必须附加说明。引用前人材料必须引证原著文字。在论文的行文上,要注意语句通顺,达到科技论文所必须具备的“正确、准确、明确”的要求。

