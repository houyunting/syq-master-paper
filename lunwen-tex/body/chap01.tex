% !TEX TS-program = XeLaTeX
% !TEX encoding = UTF-8 Unicode

\chapter{盲信号处理理论基础}
\label{chap01}

盲信号处理问题可以粗略地表述为:在一个多输入多输出(MIMO)系统中,
当传输信道特性未知或所知甚少时,从输出端的传感器阵列或转换器输出信号中,
分离或估计出源信号的波形,再或者辨识系统的数学模型。\ucite{BSS06}
根据这个粗略的定义可以看到,盲信号处理与传统信号处理的一个主要
区别是它试图仅依据某些值的最大(或最小)化来提前系统的信息。
它与神经网络中的无师学习和模式识别中的动态聚类有着紧密的联系。

盲信号处理是信号处理领域随着数字通信和地球物理勘探等行业的飞速发展兴起的一个新的研究方向。
其主要任务是对未知系统(原理、构成都未知),且对其输入信号完全未知或者仅有很少先验知识的情况下,
仅根据输出信号来重构输入信号或分析系统的原理构成等(系统辨识)。
在天线阵列处理、语音增强和分离、数字通信中码间干扰和多径效应的消除、
地球物理勘探建模(地震反卷积)、超声分析中都有着广泛的用途。\ucite{BSS06}

正因为进行源信号参数估计时缺乏混合系统和/或滤波过程参数,
盲信号处理初看起来有些不可思议,也很难想像其可以将原始信号完全恢复出来。
实际上也是如此,在缺乏足够、必要的先验知识时,唯一地确定源信号是不可能的,
然而如果在某些方面允许不确定性和模糊性,则可能估计出源信号的某些方面的特性。
在数学上,“这种不确定性和模糊性可以看做是对被估计的源信号的任意比例的伸缩、排序或时滞,
但它依然保留了源信号的波形信息“\ucite{BSS06}。
虽然这些问题会使盲信号处理的应用受到一定的限制,但在许多实际问题中这并不是我们最关注的问题,
我们最关注的往往是源信号的波形信息,而不是信号的振幅或者系统输出的排列顺序,
而盲信号处理的方法恰恰满足了这方面的需求。

盲信号处理的问题可以分为盲辨识、盲反卷积、盲信号分离三类。\ucite{BSS06}

\section{盲辨识}
盲辨识是盲信号处理问题中的一种,这种问题是如何仅利用系统的带噪输出来辨识系统数学模型的。
盲辨识问题与传统系统辨识问题之间一个最大的不同是:盲辨识中仅系统输出可以测量,而系统输入不可测量。
这就要求使用特别的信号处理方法、根据特定的假设或使用特定方法对系统的带噪输出进行处理,
得到其它信息,以补偿系统输入的未知。

通常使用的方法是过采样技术和高阶统计信息理论。
\paragraph*{过采样技术}
是对系统输出进行过采样,使其具有循环平稳特性的信号,并包含着系统的相位信息,
避免混叠、改善分辨率以及降低噪声,从而为系统模型的建立提供了可能。\ucite{OVERSAMPLED-ICA}

20世纪80年代初,Garder发现利用调制信号的循环平稳特性,
可应用二阶统计理论对通信信道的幅值和相位进行恢复\ucite{Gardner_1980},
从而掀起了采用二阶统计理论进行系统盲辨识研究的热潮。
其实现方法是采用过采样技术,即在输入信号的一个采样周期内,系统输出被采样若干次,
这样得到的输出信号同时包含着系统的相位信并具有循环平稳特性,
从而对系统输入信号未知这一缺陷进行了补偿,可对系统进行盲辨识。过采样技术的优势包括运算量适中,
算法收敛性好,可辨识非最小相位系统等,因而引发了广泛的关注。

自从Garder提出通信信道的幅值和相位进行恢复的二阶统计理论以来,过采样技术取得了很大的发展,
目前的研究主要集中在有限冲击响应系统脉冲序列的确定及算法的可辨识性方面。

\paragraph*{高阶统计信息理论}
是二阶相关概念的一种自然推广,其思想是以一定的方式对观测值进行多次相关。
20世纪80年代初期,Li和Rosenblatt运用高阶统计信息理论对非高斯系统进行了研究分析,
提出了对各类非高斯问题的统一描述\ucite{LII-MR},
从而使高阶统计信息理论在将近二十年来得到了突飞猛进的发展。
高阶统计信息理论不但能对系统进行辨识,而且可用于估计系统的相位,
即可以用于非最小相位系统的辨识。其主要缺点是计算累积误差大,数据窗口长,
数值计算量大,且仅适用于非高斯系统。

盲辨识问题时一个极具挑战性的研究课题,牵涉的知识面广。应用
过采样技术对系统进行盲辨识缺乏系统的理论和方法,还处于发展阶段。
而高阶统计信息理论经过多年的发展,取得了很多有益的成果,但主要还
停留在理论研究方面,工程实践应用还比较少。

\section{盲反卷积}
盲反卷积\ucite{BSS06}\ucite{SDU-BSS}是仅利用系统的带噪输出对系统的输入进行重构,一般假定
系统输入u(t)为独立、同分布、零均值序列,系统模型未知,系统的输出可以进行测量。
%          ______________                  ________
%输入u(t)  |             |系统带噪输出y(t) |       | ὒ(t)
%———————-->|未知系统模型 |---------------->|滤波器 |-------->
%          |_____________|                 |_______|

更一般地,传感器测得的信号是源信号滤波和延迟的混叠信号的线性组合,通常称为卷积混叠。
无噪声多通道卷积混叠信号的数学模型可以用下式表示:
\begin{equation} 
\bm{x}(t)=\bm{A}(z)\bm{s}(t) = 
\sum^\infty_{k=-\infty}\bm{A}(k)\bm{s}(t-k)
\end{equation}
式中,$\bm{x}(t)$和$\bm{s}(t)$同为瞬时混叠式;$\bm{A}(k)$为未知滤波混叠矩阵,
$\bm{A}(z)$为其Z变换;观测信号$\bm{x}(t)$是源信号$\bm{s}(t)$通过$\bm{A}(k)$的卷积混合,
所以矩阵序列$\{\bm{A}(k)\}$又称冲激响应。

仅通过观测信号$\bm{x}(t)$估计通道冲击响应$\{\bm{A}(k)\}$,进而恢复源信号,
是现有的大部分多通道盲反卷积方法采用的思路。盲反卷积(盲均衡)模型直接给在了下面:
\begin{equation}
\bm{y}(t) = \sum^\infty_{k=-\infty}\bm{W}(k)\bm{x}(t-k)=\bm{W}(z)\bm{x}(t)
\end{equation}
其中,$\bm{y}(t)$为均衡输出矢量;$\bm{W}(z)$称为均衡器,$\bm{W}(k)$为均衡器系数矩阵。

实际进行盲反卷积的算法有:
\paragraph*{H-J算法的扩展\ucite{Platt92networksfor}}
这类算法直接推广自瞬时混叠盲源分离H-J网络训练算法。针对反馈分离网络和卷积混叠模型,
卷积混叠信号的反馈分离网络输出为
\begin{equation}
\bm{y}(t)=\bm{x}-\sum^{L}_{k=0}\bm{W}(k)\bm{y}(t-k)
\end{equation}

Platt和Faggin确定了最小输出功率原理为这类网络进行优化的理论准则,
即信号的功率为最小时恰好独立分量得以分离,根据梯度下降法由此可以推出该网络的训练公式,
这恰好就是H-J算法的推广,公式为:
\begin{equation}
\Delta\omega_{ij}(k)=a y_i(t)y_j(t)
\end{equation}

\paragraph*{累积量算法的扩展}
这是自适应盲反卷积算法基于互累积量消失,该算法利用了盲源分离的经验公式,直接将盲源分离的情况扩展到盲反卷积。
公式在形式上类似于扩展的H-J方法,
并且以$c_{31}(y_i^3(t)y_j(t))$ 和 $c_{31}(y_i(t)y_j^3(t))$代替H-J公式中的非线性函数用四阶互累积量,
训练公式为:
\begin{equation}
\Delta\omega_{ij}(k)=-ac_{31}(y_i(t)y_j(t-k))
\end{equation}
\paragraph*{信息理论算法扩展}
Infomax方法是对瞬时混叠信号模型较成功的盲源分离方法之一。
Torkkola将Infomax算法推广到卷积混叠的情况,得到了一个局部训练算法(仅有两个源的情况)。
该算法通过因果滤波器最大化输出熵最小化两个输出之间的互信息,
可以基于最大信息传输原理得出直接滤波器零延迟权系数,直接滤波非零延迟权系数和反馈交叉滤波权系数的训练公式。
Lee给出了更一般的卷积混叠情况下反卷积系统滤波器权系数的公式\ucite{Lee97independentcomponent}:
\begin{eqnarray}
\Delta\bm{W}(0)	= \alpha [\bm{(I+uy^T)W}(0)] \\
\Delta\bm{W}(k)	= \beta [\bm{uy^T}(t-k)\bm{W}(k)]
\end{eqnarray}
Lambert发展了FIR滤波器多项式代数理论,并通过仿真实例表明该理论是解决多通道盲源分离的一种有效工具,
成果发表于在他的博士论文中\ucite{Lambert96multichannelblind}。
应用FIR多项式矩阵代数的基本思路是扩展标量矩阵代数到时域的滤波器矩阵代数(或频域的多项式矩阵)。
利用这一理论可以得到频域扩展的Infomax算法:
\begin{eqnarray}
\Delta\bm{W} = \alpha [\bm{W}^{-H}+FFT(\bm{u})\bm{X}^H]
\end{eqnarray}
其中,$H$表示复共轭,$\bm{W}$为滤波器矩阵,$\bm{X}$和$\bm{u}$
为频率域多传感器信号块。

\section{盲源分离}
所谓盲信号分离,亦即盲源分离,是在没有关于混合矩阵的任何先验
知识的情况下,来辨识混合矩阵或恢复所有源信号,它撇开了混合矩阵A的
特性,而只利用接受信号所携带的信息,这种盲方法的性能在本质上将不受
传输模型误差或阵列误差的影响。

盲源分离和所谓独立分量分析(Independent Component Analysis, ICA)
有着非常密切的联系,很多时候,两个名称甚至混用,关于此部分内容,
将在下一章详细介绍。


