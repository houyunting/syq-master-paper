% !TEX TS-program = XeLaTeX
% !TEX encoding = UTF-8 Unicode

%%%%%%%%%%%%%%%%%%%%%%%%%%%%%%%%%%%%%%%%%%%%%%%%%%%%%%%%%%%%%%%%%%%%%%
%
%	大连理工大学硕士论文 XeLaTeX 模版 —— 封面文件 cover.tex
%	版本:0.6
%	最后更新:2010.11.15
%	修改者:Yuri (E-mail: yuri_1985@163.com)
%	编译环境:Ubuntu 10.04 + TeXLive 2010 + TeXworks
%
%%%%%%%%%%%%%%%%%%%%%%%%%%%%%%%%%%%%%%%%%%%%%%%%%%%%%%%%%%%%%%%%%%%%%%

\cdegree{硕~~士~~学~~位~~论~~文}
\ctitle{基于独立成分分析的煤仓沉降数据分析}
\etitle{Data Analysis of the Settlement of Coal bulker based on Independent Component Analysis}

% 根据需要添加字符间距
\csubject{{\quad\;}大地测量学与测量工程}
\cauthor{{\quad\;}苏~运~强}
\cauthorno{{\quad\;}470920341}
\csupervisor{{\quad\;}赵~~国~~忱 教授}

% 这里默认使用最后编译的时间,也可自行给定日期,注意汉字和数字之间的空格。
\cdate{{\quad\;}\the\year~年~\the\month~月~\the\day~日}

\cabstract{
随着社会的发展,人类对能源的需求越来越大。当前我国的主要一次能源仍然是煤炭。
更加规范的管理煤炭的生产、运输、使用等在对于安全生产、保护环境等是非常重要
的。其中,煤仓是煤炭的生产、运输、使用等环节中都非常重要的设施,其可以作为各
环节之间的缓冲,也可以避免煤炭露天堆放产生的消耗和环境污染。
煤仓是一种大型的工业建筑,为了确保安全运营,为以后的设计提供原始资料等
必须要对煤仓进行定期的沉降观测。
当前有多种技术应用到了建筑物的变形监测中,其中水准方法应用是最常用的方法。
对采集到的数据需要进行整理以及采用一定的数学方法进行分析,
方法包括传统的几何分析以及20世纪70年代以后引入的考虑时间关联性的方法。
在对观测数据进行处理得同时,人们也试图对变形进行物理解释。

本文首先编写了程序来对煤仓沉降数据进行整理,生成相应的报表、绘制沉降曲线。
本文试图将当前主要用于图像特征提取、语音分离等的独立成分分析(ICA)技术应用
煤仓的沉降数据处理:将煤仓上多个不同点沉降观测序列作为多个信号,
应用独立成分分析的FastICA算法,获取与观测值对应的独立成分。
将得到的独立成分与煤仓的装载量、气温、时间等在同一图像上进行对比,
并且同时使用相关系数、互信息以及灰关联系数作为指标分析它们之间的关系。
通过分析独立成分与各因素的关系,可以看出,煤仓的装载量总是与一个独立成分
呈现明显的正相关关系,而气温、时间等并没有显示出这种明显的关联性。
同时,在数据质量不能保证的情况下,得到的独立成分会显得比较混乱,
这同时说明,独立成分分析对于数据的质量比较敏感,
可以用来确定观测数据的有效性。

粮仓、油库、水坝等与煤仓类似,同样可以使用独立成分分析的方法进行数据分析,
分析各自的变形与各影响因素之间的关系。
}

\ckeywords{煤仓;沉降观测;变形监测数据处理;独立成分分析;FastICA}

\eabstract{
With the development of society, more energy is being required. 
At present, coal is still the primary energy for China.
More standardized management of coal production, transportation, use, etc. 
is very important for the safe production and environmental protection. 
Of. Among them, coal bunkers are widely used in coal production, 
transportation and other sectors and are very important facilities, 
which can be used as the buffer between sections, 
and can also used to avoid the environmental pollution generated by open dumping of coal, 
so the coal mines, coal Sales, power plant, coal chemical plants, etc.

Coal bunker is a large industrial building, 
or even up to hundred meters in diameter, up to tens of meters tall.
Carrying capacity can reach several hundred thousand tons. 
Such a huge building, so much the load will be a great challenge to 
the design, construction and operation,
of course also to the foundation and the great test of bunker wall.

In order to ensure the safe operation of the coal bunker, 
and advising for the future design, etc.
the settlement observation on the coal bunker must be regular, 
and using some of the mathematical approach for data processing,
to analyze coal bunker settlement rules and to predict the further sinking coal bunker.

There are many techniques currently applied to the 
deformation monitoring of the building. For buildings that are not to giant,
the leveling method is still the most appropriate and commonly used method.

The data collected needs to be processed: 
traditional geometric deformation analysis includes
analysis of the stability of the reference point,
Observations of the adjustment processing and quality assessment, 
and the deformation parameter estimation, etc.;
in order to consider the deformation of the body in the time 
between the different states of relevance, 
since the 1970s, gradually introduced in time series analysis method, 
based on digital signal processing digital filtering technology component separation time,
deformation of the Kalman filter model and other methods.

At the same time with trying best to the processing of observational data, 
people are also trying to make the physical interpretation of the deformation.
And produced three main types of methods: the statistical analysis, 
to determine the function and mixed model method.

Here, first, wrote a program to process the data: 
generate reports, drawing settlement curve, etc.
Thus work efficiency was greatly improved.

This article attempts use Independent Component Analysis (ICA) 
which is mainly used on image feature extraction, 
speech separation, for coal bunker settlement data processing.
Will settlement observation sequences of several different points on the coal bunker be multiple signals, 
and apply them to independent component analysis to got several independent components.
In practice, FastICA algorithm can be used either directly or be used with principal component analysis (PCA). 
With obtain the corresponding independent component, the loading, temperature and time were compared.
With using the Pearson correlation coefficient, mutual information, 
and gray relational coefficients as indicators of the relationship between them,
some results achieved.
}

\ekeywords{Coal bulker; Settlement observation; Deformation monitoring; Independent Component Analysis; FastICA}

\makecover 
